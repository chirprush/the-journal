Here's a cute little problem that I actually had to solve while programming a
Rubik's cube in 3D. While rotating the cube in small steps, we can keep a
matrix in the background that keeps track of these rotations applied to the
cube in case we need to invert them. \MarginComment{Specifically, when rotating
individual faces of the cube, we need to invert the cube rotations, apply our
correct face rotaion around some axis, and then reapply the cube rotation. It's
pretty fun actually.} The problem with this, however, is that we gradually run
into floating point issues and other numerical differences, resulting in the
rotation applied to the cube and the accumulated rotation falling "out of sync"
so to speak.

Now I could have gone hunting for optimizations and other ways to try and
duct-tape a solution together, but really what I needed was just a fast and
simple way of getting the rotation. \MarginComment{For reference, this was done
in the CMU CS Academy\texttrademark\ portal, so the project was 1) really not
performant and 2) on a deadline.} After a short while of thinking, there's
actually a pretty nice solution.

In addition to the cube, what if we stored some extra points and looked at how
they were transformed under the rotations? If we had enough points, could we
uniquely determine our rotation \( R \)? Seeing as how I'm writing this, the
answer is indeed yes.

In particular, say we have \( 3 \) three-dimensional vectors \( \bvec{u} \), \( \bvec{v} \), and \( \bvec{w} \) along with their respective transformed vectors under our rotation \( R \): \( \bvec{u}' \), \( \bvec{v}' \), and \( \bvec{w}' \). Then we really have these three cases:
\begingroup
\newcommand{\point}[1]{\ensuremath{\begin{bmatrix} #1_1 \\ #1_2 \\ #1_3 \end{bmatrix}}}
\newcommand{\R}{\ensuremath{\begin{bmatrix} R_{1,1} & R_{1,2} & R_{1,2} \\ R_{2,1} & R_{2,2} & R_{2,3} \\ R_{3,1} & R_{3,2} & R_{3,3} \end{bmatrix}}}
\[
    R \bvec{u} = \bvec{u}' \qquad R \bvec{v} = \bvec{v}' \qquad R \bvec{w} = \bvec{w}'
\]
In order to do something with this, we first expand \( R \) as the following
and then decompose matrix equation for each point into three linear equations.
\[
    R = \R
.\]

For vectors \( \bvec{u} \), \( \bvec{v} \), and \( \bvec{w} \),

\begin{tikzpicture}[node distance=4cm]
    \node (lhs) { \(
        \begin{aligned}
            \R \point{\bvec{u}} = \point{\bvec{u}'}
        \end{aligned}
    \) };
    \node[right=of lhs] (rhs) { \(
        \begin{aligned}
             R_{1,1} \bvec{u}_1 + R_{1,2} \bvec{u}_2 + R_{1,3} \bvec{u}_3 &= \bvec{u}'_1 \\
             R_{2,1} \bvec{u}_1 + R_{2,2} \bvec{u}_2 + R_{2,3} \bvec{u}_3 &= \bvec{u}'_2 \\
             R_{3,1} \bvec{u}_1 + R_{3,2} \bvec{u}_2 + R_{3,3} \bvec{u}_3 &= \bvec{u}'_3
        \end{aligned}
    \) };
    \draw[->, semithick] (lhs.east) -- (rhs.west);
\end{tikzpicture}

\begin{tikzpicture}[node distance=4cm]
    \node (lhs) { \(
        \begin{aligned}
            \R \point{\bvec{v}} = \point{\bvec{v}'}
        \end{aligned}
    \) };
    \node[right=of lhs] (rhs) { \(
        \begin{aligned}
             R_{1,1} \bvec{v}_1 + R_{1,2} \bvec{v}_2 + R_{1,3} \bvec{v}_3 &= \bvec{v}'_1 \\
             R_{2,1} \bvec{v}_1 + R_{2,2} \bvec{v}_2 + R_{2,3} \bvec{v}_3 &= \bvec{v}'_2 \\
             R_{3,1} \bvec{v}_1 + R_{3,2} \bvec{v}_2 + R_{3,3} \bvec{v}_3 &= \bvec{v}'_3
        \end{aligned}
    \) };
    \draw[->, semithick] (lhs.east) -- (rhs.west);
\end{tikzpicture}

\begin{tikzpicture}[node distance=4cm]
    \node (lhs) { \(
        \begin{aligned}
            \R \point{\bvec{w}} = \point{\bvec{w}'}
        \end{aligned}
    \) };
    \node[right=of lhs] (rhs) { \(
        \begin{aligned}
             R_{1,1} \bvec{w}_1 + R_{1,2} \bvec{w}_2 + R_{1,3} \bvec{w}_3 &= \bvec{w}'_1 \\
             R_{2,1} \bvec{w}_1 + R_{2,2} \bvec{w}_2 + R_{2,3} \bvec{w}_3 &= \bvec{w}'_2 \\
             R_{3,1} \bvec{w}_1 + R_{3,2} \bvec{w}_2 + R_{3,3} \bvec{w}_3 &= \bvec{w}'_3
        \end{aligned}
    \) };
    \draw[->, semithick] (lhs.east) -- (rhs.west);
\end{tikzpicture}
\endgroup

Notice now that we have three relations for each three rows of coefficients in \( R \). This let's us now rearrange the nine equations we have as the following three systems of three equations, which we can again put back into matrix form.
\begingroup
\newcommand{\point}[1]{\ensuremath{\begin{bmatrix} #1_1 \\ #1_2 \\ #1_3 \end{bmatrix}}}
\newcommand{\pointsub}[2]{\ensuremath{\begin{bmatrix} #1_{#2,1} \\ #1_{#2,2} \\ #1_{#2,3}  \end{bmatrix}}}
\newcommand{\A}{\ensuremath{\begin{bmatrix} \bvec{u}_1 & \bvec{u}_2 & \bvec{u}_3 \\ \bvec{v}_1 & \bvec{v}_2 & \bvec{v}_3 \\ \bvec{w}_1 & \bvec{w}_2 & \bvec{w}_3 \end{bmatrix}}}

\begin{tikzpicture}[node distance=4cm]
    \node (lhs) { \(
        \begin{aligned}
            R_{1,1} \bvec{u}_1 + R_{1,2} \bvec{u}_2 + R_{1,3} \bvec{u}_3 &= \bvec{u}'_1 \\
            R_{1,1} \bvec{v}_1 + R_{1,2} \bvec{v}_2 + R_{1,3} \bvec{v}_3 &= \bvec{v}'_1 \\
            R_{1,1} \bvec{w}_1 + R_{1,2} \bvec{w}_2 + R_{1,3} \bvec{w}_3 &= \bvec{w}'_1
        \end{aligned}
    \) };
    \node[right=of lhs] (rhs) { \(
        \begin{aligned}
            \A \pointsub{R}{1} = \point{\bvec{u}'}
        \end{aligned}
    \) };
    \draw[->, semithick] (lhs.east) -- (rhs.west);
\end{tikzpicture}

\begin{tikzpicture}[node distance=4cm]
    \node (lhs) { \(
        \begin{aligned}
            R_{2,1} \bvec{u}_1 + R_{2,2} \bvec{u}_2 + R_{2,3} \bvec{u}_3 &= \bvec{u}'_2 \\
            R_{2,1} \bvec{v}_1 + R_{2,2} \bvec{v}_2 + R_{2,3} \bvec{v}_3 &= \bvec{v}'_2 \\
            R_{2,1} \bvec{w}_1 + R_{2,2} \bvec{w}_2 + R_{2,3} \bvec{w}_3 &= \bvec{w}'_2
        \end{aligned}
    \) };
    \node[right=of lhs] (rhs) { \(
        \begin{aligned}
            \A \pointsub{R}{2} = \point{\bvec{u}'}
        \end{aligned}
    \) };
    \draw[->, semithick] (lhs.east) -- (rhs.west);
\end{tikzpicture}

\begin{tikzpicture}[node distance=4cm]
    \node (lhs) { \(
        \begin{aligned}
            R_{3,1} \bvec{u}_1 + R_{3,2} \bvec{u}_2 + R_{3,3} \bvec{u}_3 &= \bvec{u}'_3 \\
            R_{3,1} \bvec{v}_1 + R_{3,2} \bvec{v}_2 + R_{3,3} \bvec{v}_3 &= \bvec{v}'_3 \\
            R_{3,1} \bvec{w}_1 + R_{3,2} \bvec{w}_2 + R_{3,3} \bvec{w}_3 &= \bvec{w}'_3
        \end{aligned}
    \) };
    \node[right=of lhs] (rhs) { \(
        \begin{aligned}
            \A \pointsub{R}{3} = \point{\bvec{u}'}
        \end{aligned}
    \) };
    \draw[->, semithick] (lhs.east) -- (rhs.west);
\end{tikzpicture}

Notice now that we have the same matrix on the right side. We'll call this
matrix \( A \). The great thing about this operation is that we only really
need to find the inverse of one matrix, namely \( A^{-1} \). Doing so, we
finally get a way to calculate the coefficients in \( R \) in terms of our vectors \( \bvec{u} \), \( \bvec{v} \), and \( \bvec{w} \) as such:
\[
    \pointsub{R}{1} = A^{-1} \point{\bvec{u}'}, \qquad \pointsub{R}{2} = A^{-1} \point{\bvec{v}'}, \qquad \pointsub{R}{3} = A^{-1} \point{\bvec{w}'} 
\]
\endgroup

With this, our Rubik's cube rotations now work properly \MarginComment{Okay for
the project it also actually took me a while to realize that I defined matrix
multiplication incorrectly, but we won't go into that.} and the world is saved,
with cubers and non-cubers alike able to enjoy the wonders that are rotations.
