Every so often, I'm scrolling through YouTube, and I find a cool problem in the
thumbnail that I just can't help but want to solve. Surprising as it is, some
Putnam problems are actually not as difficult as you might think, and the
calculus problems especially are always quite a bit of fun. Besides, it seems
I'm on quite the roll for cool competition math problems, so I might as well
continue on.

The problem that caught my eye goes as follows:

\begin{blackbox}
    \begin{problem}[77th Putnam Problem A3]
        Let \( f \colon \mathbb{R} \to \mathbb{R} \) be the solution to the functional equation
        \[
            f \left( x \right) + f \left( 1 - \frac{1}{x}  \right) = \arctan{x}
        .\]
        Find
        \[
            \int_{ 0 }^{ 1 } f \left( x \right) \, dx
        .\]
    \end{problem}
\end{blackbox}

\subsection{Solving the Functional Equation}

While at first this does look a bit daunting, let's play around a bit with the
expressions inside the functions, solving for some values. Let \( x = 2 \), giving us
\[
    f \left( 2 \right) + f \left( 1/2 \right) = \arctan{2}
.\]
Now we have two unknowns, so let's try and see if there's another relation involving \( f \left( 1/2 \right) \). Let's set \( x = 1/2 \), giving us:
\[
    f \left( 1/2 \right) + f \left( -1 \right) = \arctan{\left( 1/2 \right)}
.\]
Let's go one level deeper, setting \( x = -1 \).
\[
    f \left( -1 \right) + f \left( 2 \right) = \arctan{\left(-1 \right)}
.\]
Quite interestingly, we see that we have three relations for three different
unknowns, which means that we can solve for each of them. What's more useful
for us though, is that this choice of \( 2 \) was arbitrary, meaning that we
can do this for any value of \( x \) (disregarding singularities).
\MarginComment{For many functional equations, simply playing around with
different substitutions or specific values often helps one gain a good insight
as to what's going on.}

This means that there is a quite simple way of solving for the function, and all we need to do is just take the approach from above but generalize it.

In particular, define \( g \left( x \right) = 1 - 1 / x \). Notice the
following very nice properties that emerge when we take iterated applications
of \( g \left( x \right) \).
\begin{align*}
    g \left( x \right) &= 1 - 1/x \\
    g \left( g \left( x \right) \right) &= 1 / \left( 1 - x \right) \\
    g \left( g \left( g \left( x \right) \right) \right) &= x
.\end{align*}
Any subsequent applications will go back to one of these three functions due
to, \( x \) being the identity function. Using this, we can now solve the
functional equation. We shall use our given equation, writing it in terms of
our new function \( g \left( x \right) \) and also the modified function
equations with the substitutions \( x \mapsto g \left( x \right) \) and \( x
\mapsto g \left( g \left( x \right) \right) \) to yield a system of three
equations with three unknowns.
\[
\begin{array}{ccccccl}
    f \left( x \right) & + & f \left( g \left( x \right) \right) & + & 0 & = & \arctan{x} \\
    0 & + &  f \left( g \left( x \right) \right) & + & f \left( g \left( g \left( x \right) \right) \right) & = & \arctan{\left( g \left( x \right) \right)} \\
    f \left( x \right) & + & 0 & + & f \left( g \left( g \left( x \right) \right) \right) & = & \arctan{\left( g \left( g \left( x \right) \right) \right)}
\end{array}
\]
Through some rather simple algebra, we can solve the system to see that
\[
    2 f \left( x \right) = \arctan{x} - \arctan{\left( g \left( x \right) \right)} + \arctan{\left( g \left( g \left( x \right) \right) \right)}
.\]
With this, we're a whole lot closer to getting our answer. Now all that's left is to integrate!

\subsection{Integrating Some Arctangents}

If we integrate on both sides of our solved functional equation, we can split
up each term into separate integrals and tackle each of them one-by-one.
\[
    2 \int_0^1 f \left( x \right) \, dx = \overset{\textbf{1.}}{\boxed{\int_{ 0 }^{ 1 } \arctan{x} \, dx}} - \overset{\textbf{2.}}{\boxed{\int_{ 0 }^{ 1 } \arctan{\left( 1 - \frac{1}{x} \right)} \, dx}} + \overset{\textbf{3.}}{\boxed{\int_{ 0 }^{ 1 } \arctan{\left( \frac{1}{1 - x} \right)} \, dx}}
.\]
We'll start off with the easiest one, which is just the integral of plain \( \arctan{x} \), which lends itself well to integration by parts:
\begin{flalign*}
    \textbf{1.} \int_{ 0 }^{ 1 } \arctan{x} \, dx &= \bounds{x \arctan{x}}{0}{1} - \int_{ 0 }^{ 1 } \frac{x}{x^2 + 1} \, \tikzmarknode{ibp1}{dx} & \\
    &= \frac{\pi}{4} - \bounds{\frac{1}{2} \ln{\abs{x^2 + 1}}}{0}{1} \tikzmarknode{usub1}{} & \\
    &= \boxed{\frac{\pi}{4} - \frac{1}{2} \ln{2}}. &
\end{flalign*}
Luckily, \( \arctan{x} \) is a simple inverse function, so there's not much
difficulty in using integration by parts to find its integral. This next
integral won't be the cleanest.
\begin{tikzpicture}[overlay, remember picture]
    \draw[<-, semithick] (ibp1) ++ (0.5, 0) -- ++ (2, 0) node[above left] {\footnotesize IBP} node[right, draw=black] {
    \(
    \footnotesize
        \arraycolsep=1pt
        \begin{array}{cclcccl}
            u & = & \arctan{x} & \  & du & = & dx /( x^2 + 1 ) \\
            dv & = & dx & \  & v & = & x \\
        \end{array}
    \)};

    \draw[<-, semithick] (usub1) ++ (0.2, 0) -- ++ (3.5, 0) node[above left] {\footnotesize substitution} node[right, draw=black] {
    \(
    \footnotesize
        \arraycolsep=1pt
        \begin{array}{ccl}
            w & = & x^2 + 1 \\
            dw & = & 2x \, dx
        \end{array}
    \)};
\end{tikzpicture}

For this integral, \MarginComment{TODO: Add in the boxes showing substitutions
and such. They take a bit of work to put in, but it makes stuff easier to
follow.} we'll again start with integration by parts, but a little more work is
required to get it into a nice form.
\begin{align*}
    \textbf{2.} \int_{ 0 }^{ 1 } \arctan{\left( 1 - \frac{1}{x} \right)} \, dx &= \bounds{x \arctan{\left( 1 - \frac{1}{x} \right)}}{0}{1} - \int_{ 0 }^{ 1 } \frac{1}{x + x\left( 1 - \frac{1}{x} \right)^2} \, dx
\end{align*}
For the left term, we can plug in \( x = 1 \) to get \( 0 \) for the top bound, but we must take the limit as \( x \to 0^+ \) for the second.
\begin{align*}
    \bounds{x \arctan{\left( 1 - \frac{1}{x} \right)}}{0}{1} &= - \lim_{x \to 0^+} x \arctan{\left( 1 - \frac{1}{x} \right)} \\
    &= - \lim_{x \to 0^+} \frac{\arctan{\left( 1 - 1/x \right)}}{1 / x} \\
    &= - \lim_{x \to 0^+} \frac{1}{1 + \left( 1 - \frac{1}{x} \right)^2} \\
    &= - \lim_{u \to -\infty} \frac{1}{1 + u^2} \\
    &= 0
.\end{align*}
We now return to our integral and work it into a nicer form:
\begin{align*}
    &- \int_{ 0 }^{ 1 } \frac{1}{x + x \left( 1 - \frac{1}{x} \right)^2} \, dx = - \int_{ 0 }^{ 1 } \frac{x}{x^2 + \left( x - 1 \right)^2} \, dx \\
    &= - \int_{ 0 }^{ 1 } \frac{x}{2x^2 - 2x + 1} \, dx = - \int_{ 0 }^{ 1 } \frac{2x}{\left( 2x - 1 \right)^2 + 1} \, dx \\
    &= - \int_{ 0 }^{ 1 } \frac{\left( 2x - 1 \right) + 1}{\left( 2x - 1 \right)^2 + 1} \, dx = -\frac{1}{2} \int_{ -\pi/4 }^{ \pi/4 } \left( \tan{\theta} + 1 \right) \, d\theta \\
    &= \boxed{\frac{-\pi}{4}}
\end{align*}
On to the next integral!

For this one, we'll first do a \( u \)-substitution of the entire integrand,
getting us to something perhaps a little bit more familiar looking.
\begin{align*}
    \textbf{3.} &\int_{ 0 }^{ 1 } \arctan{\left( \frac{1}{1 - x} \right)} \, dx \\
    &= \int_{\pi/4}^{\pi/2} u \csc^2{u} \, du
.\end{align*}
We'll now do just a little bit of tabular integration:
\[
\begin{array}{ccc}
    & D & I \\
    + & u & \csc^2{u} \\
    - & 1 & -\cot{u} \\
    + & 0 & -\ln{\abs{\sin{u}}}
\end{array}
\]
This means that the integral becomes
\begin{align*}
    &\left[ -u\cot{u} + \ln{\abs{\sin{u}}} \rule{0pt}{10pt} \right]_{\pi/4}^{\pi/2} \\
    &= \boxed{\frac{\pi}{4} + \frac{1}{2} \ln{2}}
,\end{align*}
which does indeed look suspiciously familiar to some of our other results.

With this, we're pretty much complete with the problem, with all that's left to
do being to substitute our values in. Taking great care to remember the \( 2 \)
we placed in front of the integral way back last section, we see that the final
answer to the problem is
\[
    \int_{0}^{1} f \left( x \right) \, dx = \frac{1}{2} \left( \frac{\pi}{4} - \frac{1}{2} \ln{2} + \frac{\pi}{4} + \frac{\pi}{4} + \frac{1}{2} \ln{2} \right) = \boxed{\frac{3 \pi}{8}}
.\]

Perhaps I'll do more Putnam problems in the future. They seem to be an
interesting intersection of higher level mathematics (including stuff like
calculus, abstract algebra, and combinatorics among others), but they
definitely seem to have a different sort of "flavor" from the competition
problems that I'm used to. Certainly, compared to the IMO and such, there are
definitely some easier problems, but I'm sure not all of them are as easy.
Perhaps I'll get to tackle the Putnam in person myself one day. Who knows?
