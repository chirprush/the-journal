Combinatorics is a really beautiful subject \MarginComment{I'd probably say
that in competition math, combinatorics is up there for my favorite subject} in
that, in premise, it can be very simple, but in practice it actually takes a
decent bit of thinking and cleverness sometimes. Being so widely defined as the
science of counting "things," combinatorics also lends itself very well to
intersections with other subjects. You can have geometry, number theory,
algebra, probability, and many others mixed into a combinatorics problem,
making them really fun and sometimes quite the challenge.

While it certainly is fun to solve these problems, I think it's also perhaps
quite instructive for me (and perhaps the readers) to create my own
combinatorics problems and gather a wide variety of practice problems. Creating
my own problems will be the perfect opportunity to get more acquainted with
some of the cool tricks and structure to problems, helping me in furthering my
combinatorics adventure and giving me some good exercise.
\MarginComment{Calisthenics is a category of exercises which rely primarily on
gravity and body weight. I thought it would be a funny name for this entry
mainly because my friends have roped me into it.}

In the process of making these, I'll probably put down whatever comes to mind
really, so some problems might be really easy or perhaps insanely cumbersome to
work with; nevertheless, it will still be a nice catalogue of my journey
through the world of counting.

As for solutions, I'll probably write down some of the cooler ones whenever I
have time, but I don't think I'll be able to get to all of the problems.

I hope to include not only competition math "flavored" problems which have slick
combinatorial arguments, but also some more exploratory problems perhaps using
generating functions and more of the higher level combinatorics stuff. With
that said, let's get to problem synthesizing!

\subsection{General Combinatorics}

\textit{Problems}: 
\begin{enumerate}
    \item How many partitions of the set \( \set{1, 2, \ldots, 20} \) have at
        least \( 10 \) elements?

    \item Exactly \( 12 \) of your homies, \MarginComment{For those with \( n <
        12 \), where \( n \) denotes the number of homies you have, you may
        borrow some for the sake of mathematics.} denoted \( h_1, h_2,\ldots,
        h_{12} \), are arranged in a circle. Starting from the first homie, \(
        h_1 \), and going in order, each homie pulls a string from themselves
        to another homie, so long as that string does not pass over any other
        strings in the process. All connections between homies must be made
        inside the circle. This process is continued until no more connections
        can be made. How many end states are there?
\end{enumerate}

\subsection{Geometry}

\textit{Problems}: 
\begin{enumerate}
    \item Suppose we have \( n \) circles in the plane. \MarginComment{This
        could potentially be infinite or trivial for large \( n \), but for \(
        k = 2 \), I at least know there's some interesting stuff to play around
        with.} Allowing for the adjustment of position and size of all circles, what is
        the maximum number of lines one can place that are tangent to \( k \) circles?
\end{enumerate}

\subsection{Number Theory}

\textit{Problems}: 
\begin{enumerate}
    \item How many pairs of positive integers \( \left( a, b, c \right) \) satisfy
        \[
            a + b + c = 23 \quad \text{and} \quad a^2 + 2b = 40
        ?\]

    \item How many partitions of the set \( \set{1, 2, \ldots, 50} \) can one
        make where the least common multiple of all elements in the partition
        is \( 50 \)?
\end{enumerate}

\subsection{Algebra}

\textit{Problems}: 
\begin{enumerate}
    \item How many polynomials of the form
        \[
            P \left( x \right) = x^2 + ax + b
        \]
        have at least one real root, for \( a, b \in \set{1, 2, \ldots, 50} \)?
\end{enumerate}

\subsection{Colorings}

\textit{Problems}: 
\begin{enumerate}
    \item (2019 CMIMC Combinatorics \#2) How many ways are there to color the
        vertices of a cube either red, blue, or green such that no two adjacent
        vertices are the same color?

    \item (2019 AMC 12A \#13) How many ways are there to paint each of the
        integers \( 2,3,\ldots,9 \) either red, green, or blue so that each
        number has a different color from each of its proper divisors?
\end{enumerate}
