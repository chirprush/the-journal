Oh boy it has been quite a while since I've been able to write in the math
journal, hasn't it? Junior year is going decently swimmingly but the work load
is certainly not as merciful and liberal as last year unfortunately. Even
still, I can probably find the time to come back here every now and again, but
I might not \MarginComment{Although, I think my writing skills will certainly
advance as the year goes on so perhaps I may turn to this for creative writing
practice.} be able to write long entries too often.

As such, I've thought it would be a nice idea to have a place to curate some
less developed but perhaps expandable and fun math thoughts that I have
throughout the day. This way, I won't get as rusty and I can get into the habit
of writing some things down. These can also serve as things I can turn into
full fledged journal entries. Fun!

\begin{itemize}
    \item It would be fun to experiment and see what breaks in probability
        theory if we were to allow for negative pdfs. Leaving physical
        interpretation aside, it may have some sort of use. In general, I do
        want to explore measure theoretic and formal probability theory more,
        but it might take a while to learn.
        
    \item This probably isn't useful but suppse \( P(x) \) is a polynomial of
        degree \( n \) with all coefficients in \( [0, 1] \). We then have
        \[
            P(x)^2 \leqslant n \frac{x^{2n + 2} - 1}{x^2 - 1}
        .\]

    \item A while ago when I was sick, I was exploring orthogonal
        polynomials/other functions in an effort to answer a (rather
        subjective) question on whether sound necessarily had to be like it is
        in real life \MarginComment{As a side note, I do also want to write a
        whole exposition on the math behind music and sound perhaps for an
        orchestra project.}. Certainly for polynomials I don't think they play well
        enough compared to waves in order to them to work, but it was a shot.
        Perhaps some other family of functions may work.
        \begin{blackbox}
            For any infinite set of orthogonal functions \( \mathcal{O} = \{ P_0 (x), P_1(x), \ldots \} \) and any function \( f \),
            \[
                f(x) \approx \sum_{n = 0}^\infty \frac{a_n}{w_n} P_n (x)
            ,\]
            where
            \begin{align*}
                &a_n = \int_{-L}^{L} f(t) P_n (t) \, dt,
                & w_n = \int_{-L}^{L} \left( P_n (t) \right)^2 \, dt
            \end{align*}
        \end{blackbox}

    \item I really should turn the (semi) gradient noise program into a journal
        entry because it's simple math that turns into cool looking art. That
        being said, it isn't really true gradient noise.

    \item Speaking of things to put in the journal, the baby derangements
        problem from stats class is like already done \MarginComment{Note:
        include the limit Max talked about because its funny.} and ready to
        just be like tweaked and put in. I really just need to get around to
        doing it.

    \item I really want to get into inequalities and formal set theory more
        because ARML has shown me that they're really cool. In general, I think
        it'd also be nice to do a lot more proof style stuff in the journal
        because I'm lacking experience with the rigor.

    \item I should look back at the problem of finding "closed forms" for
        \[
            \sum_{k = 0}^{\infty} \frac{x^k}{(nk)!}
        .\]
        I believe it would also be nice to show binomial multiple sums
        \MarginComment{Future me will probably know what this is talking about.}
        using a similar technique.

    \item I really excited to write about the restricted random walk problem I
        showed Lester as it has some really cool physical implications; I just
        need the time. This is so sad. (As a side note I really like how I set
        up the \LaTeX\ formatting for that so in case I ever need something to
        steal I might as well go with that).

    \item I should get back to the graph theory stuff sometime. Likely the
        thing that will keep me interested is the one neighbor derangements
        problem.

    \item Let
        \[
            Y = \sum_{k = 1}^{N} X_k
        ,\]
        where \( X_1, X_2, \ldots, X_N \sim \textsf{Uniform}(0, r) \) are i.i.d. The probabilitiy density function for \( Y \) is
        \[
            f_Y (t) = \frac{1}{r^n (n-1)!} \sum_{k = 0}^{N} \binom{N}{k} (-1)^k (t - rk)^{N - 1} H(t - rk)
        ,\]
        where \( H(t) \) denotes the Heaviside step function.

    \item Oh yeah I just remembered the one simplex probability problem journal
        entry still exists. Hopefully that gets completed by the end of the
        year.

    \item Suppose \( X_0 \sim \textsf{Uniform} (0, 1) \) and \( X_n \sim \textsf{Uniform} (0, X_0) \). Then \( \mathrm{E} (X_n) = 1/n! \).
\end{itemize}
