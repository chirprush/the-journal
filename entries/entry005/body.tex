I'd like to think that this is a decently interesting question that stems from
some other stuff I've worked with in differential equations, we sometimes have
ODEs of the form
\[
    a_0y + a_1y' + \cdots + a_n y^{\left( n \right)} = 0
.\]
These are readily solveable using a method called the \textit{characteristic
polynomial}, \MarginComment{As we'll see later, there will be a few analogues
to these differential equations and characteristic polynomials in this
solution. Connections in math are fun!} which I won't get into too much here
(but perhaps another time?), where essentially we just assume an exponential
solution utilizing the roots of a similar looking polynomial.

Here's where sort of my inspiration comes in. Instead of looking at
differential equations that have characteristic polynomials or potential
polynomials in \( x \), what if we look at differential equations that are
\textit{polynomials in \( y \)}?

\subsection{Polynomials in \( y \)}

\begin{blackbox}
    \begin{problem}
        What can we say about differential equations of the following form and
        their solutions?
        \[
            \frac{dy}{dx} = a_0 + a_1 y + \cdots + a_n y^n
        .\]
    \end{problem}
\end{blackbox}

Because the right side is written completely in terms of \( y \), we can
recognize that this is simply a first order separable differential equation.
With this, we can move the polynomial to the left hand side by dividing and
then simply just integrate with respect to y. Then our problem lies in finding
out how to integrate this general rational function. Now where have we heard
that?

Using the same result, \MarginComment{I was quite excited when I derived it
myself, so I've used it quite a bit in various places. It's quite fun.} from
the previous journal entry, we can integrate this rational function assuming we
know the roots of the polynomial in \( y \), which we will denote as \( y_1,
y_2, \ldots, y_n \). Thus we have that
\[
    \frac{1}{a_n \left( y - y_1 \right) \left( y - y_2 \right) \cdots \left( y - y_n \right)} \cdot \frac{dy}{dx} = 1
.\]
