\newcommand{\boxdiagram}[7][]{
    \begin{center}
    \begin{tikzpicture}
        \draw[thick] (0, 0) rectangle (1, 1);
        \draw[thick] (2, 0) rectangle (3, 1);
        \draw[thick] (4, 0) rectangle (5, 1);
        \draw[thick] (6, 0) rectangle (7, 1);
        \draw[thick] (8, 0) rectangle (9, 1);
        \draw[thick] (10, 0) rectangle (11, 1);

        \node at (0.5, 0.5) {#2};
        \node at (2.5, 0.5) {#3};
        \node at (4.5, 0.5) {#4};
        \node at (6.5, 0.5) {#5};
        \node at (8.5, 0.5) {#6};
        \node at (10.5, 0.5) {#7};

        \ifthenelse{\equal{#1}{}}{}{
            \draw[thick, ->] (5.5, -0.25) -- (5.5, -1.25);
            \node at (6, -0.75) {\ensuremath{#1}};
        }
    \end{tikzpicture}
    \end{center}
}

Hay hay! It's the summer, so I actually have quite a bit more time now to spend
on problems and such. \MarginComment{I'm actually really excited! I've vowed to
be more productive over the summer so this seems like the perfect opportunity,
although I do have to self-study some stuff.} With this extra time, I thought
it would be really cool to try and tackle a far harder competition math problem
than usual. After looking through a couple problems, I chanced upon a very cool
looking IMO problem, so I thought it would be perfect for the occasion. Here's
how it goes:

\begin{blackbox}
    \begin{problem}[2010 IMO Problem \#5]
        Each of the six boxes \( B_1, B_2, B_3, B_4, B_5, B_6 \) initially
        contains one coin. The following operations are allowed:

        \begin{enumerate}[topsep=10pt, itemsep=5pt]
            \item Choose a non-empty box \( B_j \), \( 1 \leqslant j \leqslant 5 \), remove
                one coin from \( B_j \) and add two coins to \( B_{j + 1} \).
            \item Choose a non-empty box \( B_k \), \( 1 \leqslant k \leqslant 4 \), remove
                one coin from \( B_k \) and swap the contents (maybe empty) of
                the boxes \( B_{k+1} \) and \( B_{k+2} \).
        \end{enumerate}

        Determine if there exists a finite sequence of operations of the
        allowed types, such that the five boxes \( B_1, B_2, B_3, B_4, B_5 \)
        become empty, while box \( B_6 \) contains exactly \(
        2010^{2010^{2010}} \) coins.
    \end{problem}
\end{blackbox}

As a brief remark on notation, I'll represent first move type by \( D_k \) and the second by \( S_k \), representing "double" and "swap" respectively, with \( k \) denoting the chosen box that the move acts upon.

I chose this problem in particular because, as you might see, there isn't any
crazy algebra or geometry or number theory going on where I would have to
recall some very specific theorem and have a great deal of intuition for the
subject (not to harp on those problems \textit{too much}; I just wouldn't be
able to solve them very well haha). This problem is quite nice in that it
requires, at least on its face, no mathematical background or knowledge but
rather a determined enough problem solver. It's a fun puzzle, and I really would
like to tackle it. So, let's get started!

\subsection{Observations}

It's time to brainstorm how to actually start the problem and sort of what I
think we're really aiming for. \MarginComment{To be honest, starting the problem
is probably one of my weak spots, although that may be the case somewhat for
everyone.} This problem is really a game with a set of rather simple moves, and
I get the feeling that this is going to be solved through pure logic rather
than any fancy or flashy math theorem. 

Because this is an IMO problem, I'm of the opinion that the answer is one of
two things:
\begin{enumerate}
    \item There \textit{does} exist such a sequence and it's rather hard to
        find.
    \item There \textit{doesn't} exist such a sequence, and the proof for such
        is rather hard.
\end{enumerate}
So, in summary, the problem is hard, but I mean that's why I picked it. I'm
leaning towards there being such a sequence because that seems a lot more fun,
but we probably wouldn't have to construct such a sequence anyways.

The proof may also revolve around some very light number theory in terms of
parity. We might also have to take into consideration ideas such as the maximum
possible number of coins to exist in the system and comparing that to our big
constant.

In fact, on the topic of The Big Constant\texttrademark, I find it somewhat
hard to imagine it being special or completely unique in some way that allows
this problem to work. \MarginComment{I mean if it is though that's pretty
amazing.} It's likely that, if the sequence does indeed exist, we're looking
for a more general way to reach a state of say \( 1, 1, 1, 1, 1, n \) or
something similar rather than the exact number.

Another thing to consider is that the first box may be some place to start as
there is only one possible move you can ever do on in throughout the course of
the game. When we do this move, the box ends up with \( 0 \) coins, and there
is no way to add back to that (and obviously we have to do a move on this box
because it has to become empty at some point.).

Another thing to consider is the associativity and the commutativity of the
moves. While I'm sure this problem could be framed entirely in a group theory or
abstract algebra setting, there isn't all too much to gain from doing so, but
taking some ideas over certainly can't hurt.

\subsection{Odds and Ends}

Let's start with the following observation: \( 2010^{2010^{2010}} \) is even,
and \( 1 \) is not. It sounds quite obvious, but notice that the only way we
can add pieces is with the doubling move, which adds \( 2 \) coins to the next
box. This means that the parity of the boxes stays the same in this case, which
we certainly don't want. We can address this then \MarginComment{The good 'ol
Canadian shuffle, as they call it.} by subtracting one, which does change the
parity, and then swapping the boxes around to get it into the desired position
as such:

\vspace{0.3cm}

\boxdiagram[D_5]{1}{1}{1}{1}{1}{1}
\boxdiagram[S_4]{1}{1}{1}{1}{0}{3}
\boxdiagram     {1}{1}{1}{0}{3}{0}

\let\boxdiagram\undefined
